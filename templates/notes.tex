When generating uniformly distributed points on a sphere using Three.js, one effective approach is to use an icosahedron-based method.
 An icosahedron is a polyhedron with 20 triangular faces and 12 vertices, providing a nearly symmetrical structure that can be subdivided to
  approximate a sphere. In Three.js, `THREE.IcosahedronGeometry(radius, subdivisions)` leverages this structure, where `radius` defines 
  the sphere's radius, and `subdivisions` specifies how many times each triangular face should be divided. With each subdivision, 
  each face of the icosahedron splits into four smaller triangles, adding vertices to the geometry. These vertices are then normalized 
  (moved to lie on the surface of the sphere by scaling them to match the sphere's radius) to form a more evenly distributed point set 
  on the sphere’s surface. The result is a dense, approximately uniform point distribution on the spherical surface without randomness.

The subdivision process works as follows: at each subdivision level, new vertices are introduced in the middle of each edge 
of the original triangles, and each new face is split into four smaller triangles. After the splitting process, each vertex coordinate 
is normalized to match the desired radius. This means each vertex is adjusted so that its distance from the origin is equal to the specified 
radius of the sphere. The normalization of vertices onto the spherical surface is essential for achieving a close approximation to a 
uniform distribution, as it avoids clustering near the edges and instead distributes points evenly. 
The normalized vertices are stored in the geometry’s `vertices` array, each representing a nearly uniform point on the sphere’s surface.


In the context of a BIM (Building Information Modeling) model, each vertex on a sphere 
 can represent a direction vector. Imagine each vertex as an arrow originating from a central point (One grid point).
This arrow extends in the direction of the vertex coordinates, creating a "ray" that can be used to detect or intersect objects 
within the BIM model.

To leverage this for object detection, we treat each vertex as a unit vector defining a direction. 
From a chosen grid point— a specific point in the model— we can create rays that extend along each 
of these directions. As these rays move outward, they intersect with various elements in the BIM model, 
such as walls, windows, or structural components. Using raycasting techniques, it is possible to detect
 which objects each ray encounters and determine the order, distance, and angle of intersection.